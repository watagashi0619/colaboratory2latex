\section{Charting in Colaboratory}
A common use for notebooks is data visualization using charts. Colaboratory makes this easy with several charting tools available as Python imports.\\
\subsection{Matplotlib}
\href{http://matplotlib.org/}{Matplotlib} is the most common charting package, see its \href{http://matplotlib.org/api/pyplot\_api.html}{documentation} for details, and its \href{http://matplotlib.org/gallery.html#statistics}{examples} for inspiration.\\
\subsubsection{Line Plots}

\columnratio{0.09}
\begin{paracol}{2}
\smallskip
\begin{cellExecute}[escapechar=~]
~\inputPrompt{1}~
\end{cellExecute}
\switchcolumn
\begin{codeCell}[escapechar=~]
~\textcolor{mtk17}{import}~ matplotlib.pyplot ~\textcolor{mtk17}{as}~ plt

x  = [~\textcolor{mtk7}{1}~, ~\textcolor{mtk7}{2}~, ~\textcolor{mtk7}{3}~, ~\textcolor{mtk7}{4}~, ~\textcolor{mtk7}{5}~, ~\textcolor{mtk7}{6}~, ~\textcolor{mtk7}{7}~, ~\textcolor{mtk7}{8}~, ~\textcolor{mtk7}{9}~]
y1 = [~\textcolor{mtk7}{1}~, ~\textcolor{mtk7}{3}~, ~\textcolor{mtk7}{5}~, ~\textcolor{mtk7}{3}~, ~\textcolor{mtk7}{1}~, ~\textcolor{mtk7}{3}~, ~\textcolor{mtk7}{5}~, ~\textcolor{mtk7}{3}~, ~\textcolor{mtk7}{1}~]
y2 = [~\textcolor{mtk7}{2}~, ~\textcolor{mtk7}{4}~, ~\textcolor{mtk7}{6}~, ~\textcolor{mtk7}{4}~, ~\textcolor{mtk7}{2}~, ~\textcolor{mtk7}{4}~, ~\textcolor{mtk7}{6}~, ~\textcolor{mtk7}{4}~, ~\textcolor{mtk7}{2}~]
plt.plot(x, y1, label=~\textcolor{mtk25}{"line L"}~)
plt.plot(x, y2, label=~\textcolor{mtk25}{"line H"}~)
plt.plot()

plt.xlabel(~\textcolor{mtk25}{"x axis"}~)
plt.ylabel(~\textcolor{mtk25}{"y axis"}~)
plt.title(~\textcolor{mtk25}{"Line Graph Example"}~)
plt.legend()
plt.show()
\end{codeCell}
\end{paracol}

\columnratio{0.09}
\begin{paracol}{2}
\begin{cellExecute}[escapechar=~]
~\outputPrompt{1}~
\end{cellExecute}
\switchcolumn
\begin{resultCell}[escapechar=~]
\end{resultCell}
\end{paracol}

\begin{figure}[H]
\centering
\includegraphics[keepaspectratio,width=0.4\hsize]{/Users/watagashi0619/colaboratoryfigures/fig191125064527_4.png}
\end{figure}
\subsubsection{Bar Plots}

\columnratio{0.09}
\begin{paracol}{2}
\smallskip
\begin{cellExecute}[escapechar=~]
~\inputPrompt{2}~
\end{cellExecute}
\switchcolumn
\begin{codeCell}[escapechar=~]
~\textcolor{mtk17}{import}~ matplotlib.pyplot ~\textcolor{mtk17}{as}~ plt

~\mtkHash{mtk8}~~\textcolor{mtk8}{ Look at index 4 and 6, which demonstrate overlap}~~\textcolor{mtk8}{ping cases.}~
x1 = [~\textcolor{mtk7}{1}~, ~\textcolor{mtk7}{3}~, ~\textcolor{mtk7}{4}~, ~\textcolor{mtk7}{5}~, ~\textcolor{mtk7}{6}~, ~\textcolor{mtk7}{7}~, ~\textcolor{mtk7}{9}~]
y1 = [~\textcolor{mtk7}{4}~, ~\textcolor{mtk7}{7}~, ~\textcolor{mtk7}{2}~, ~\textcolor{mtk7}{4}~, ~\textcolor{mtk7}{7}~, ~\textcolor{mtk7}{8}~, ~\textcolor{mtk7}{3}~]

x2 = [~\textcolor{mtk7}{2}~, ~\textcolor{mtk7}{4}~, ~\textcolor{mtk7}{6}~, ~\textcolor{mtk7}{8}~, ~\textcolor{mtk7}{10}~]
y2 = [~\textcolor{mtk7}{5}~, ~\textcolor{mtk7}{6}~, ~\textcolor{mtk7}{2}~, ~\textcolor{mtk7}{6}~, ~\textcolor{mtk7}{2}~]

~\mtkHash{mtk8}~~\textcolor{mtk8}{ Colors: }~~\textcolor{mtk8}{https://matplotlib.org/api/colors}~~\mtkTilde{mtk8}~~\textcolor{mtk8}{api.html}~

plt.bar(x1, y1, label=~\textcolor{mtk25}{"Blue Bar"}~, color=~\textcolor{mtk25}{'b'}~)
plt.bar(x2, y2, label=~\textcolor{mtk25}{"Green Bar"}~, color=~\textcolor{mtk25}{'g'}~)
plt.plot()

plt.xlabel(~\textcolor{mtk25}{"bar number"}~)
plt.ylabel(~\textcolor{mtk25}{"bar height"}~)
plt.title(~\textcolor{mtk25}{"Bar Chart Example"}~)
plt.legend()
plt.show()
\end{codeCell}
\end{paracol}

\columnratio{0.09}
\begin{paracol}{2}
\begin{cellExecute}[escapechar=~]
~\outputPrompt{2}~
\end{cellExecute}
\switchcolumn
\begin{resultCell}[escapechar=~]
\end{resultCell}
\end{paracol}

\begin{figure}[H]
\centering
\includegraphics[keepaspectratio,width=0.4\hsize]{/Users/watagashi0619/colaboratoryfigures/fig191125064527_6.png}
\end{figure}
\subsubsection{Histograms}

\columnratio{0.09}
\begin{paracol}{2}
\smallskip
\begin{cellExecute}[escapechar=~]
~\inputPrompt{3}~
\end{cellExecute}
\switchcolumn
\begin{codeCell}[escapechar=~]
~\textcolor{mtk17}{import}~ matplotlib.pyplot ~\textcolor{mtk17}{as}~ plt
~\textcolor{mtk17}{import}~ numpy ~\textcolor{mtk17}{as}~ np

~\mtkHash{mtk8}~~\textcolor{mtk8}{ Use numpy to generate a bunch of random data in }~~\textcolor{mtk8}{a bell curve around 5.}~
n = ~\textcolor{mtk7}{5}~ + np.random.randn(~\textcolor{mtk7}{1000}~)

m = [m ~\textcolor{mtk17}{for}~ m ~\textcolor{mtk6}{in}~ ~\textcolor{mtk13}{range}~(~\textcolor{mtk13}{len}~(n))]
plt.bar(m, n)
plt.title(~\textcolor{mtk25}{"Raw Data"}~)
plt.show()

plt.hist(n, bins=~\textcolor{mtk7}{20}~)
plt.title(~\textcolor{mtk25}{"Histogram"}~)
plt.show()

plt.hist(n, cumulative=~\textcolor{mtk6}{True}~, bins=~\textcolor{mtk7}{20}~)
plt.title(~\textcolor{mtk25}{"Cumulative Histogram"}~)
plt.show()
\end{codeCell}
\end{paracol}

\columnratio{0.09}
\begin{paracol}{2}
\begin{cellExecute}[escapechar=~]
~\outputPrompt{3}~
\end{cellExecute}
\switchcolumn
\begin{resultCell}[escapechar=~]
\end{resultCell}
\end{paracol}

\begin{figure}[H]
\centering
\includegraphics[keepaspectratio,width=0.4\hsize]{/Users/watagashi0619/colaboratoryfigures/fig191125064527_8.png}
\end{figure}
\subsubsection{Scatter Plots}

\columnratio{0.09}
\begin{paracol}{2}
\smallskip
\begin{cellExecute}[escapechar=~]
~\inputPrompt{4}~
\end{cellExecute}
\switchcolumn
\begin{codeCell}[escapechar=~]
~\textcolor{mtk17}{import}~ matplotlib.pyplot ~\textcolor{mtk17}{as}~ plt

x1 = [~\textcolor{mtk7}{2}~, ~\textcolor{mtk7}{3}~, ~\textcolor{mtk7}{4}~]
y1 = [~\textcolor{mtk7}{5}~, ~\textcolor{mtk7}{5}~, ~\textcolor{mtk7}{5}~]

x2 = [~\textcolor{mtk7}{1}~, ~\textcolor{mtk7}{2}~, ~\textcolor{mtk7}{3}~, ~\textcolor{mtk7}{4}~, ~\textcolor{mtk7}{5}~]
y2 = [~\textcolor{mtk7}{2}~, ~\textcolor{mtk7}{3}~, ~\textcolor{mtk7}{2}~, ~\textcolor{mtk7}{3}~, ~\textcolor{mtk7}{4}~]
y3 = [~\textcolor{mtk7}{6}~, ~\textcolor{mtk7}{8}~, ~\textcolor{mtk7}{7}~, ~\textcolor{mtk7}{8}~, ~\textcolor{mtk7}{7}~]

~\mtkHash{mtk8}~~\textcolor{mtk8}{ Markers: }~~\textcolor{mtk8}{https://matplotlib.org/api/markers}~~\mtkTilde{mtk8}~~\textcolor{mtk8}{api.html}~

plt.scatter(x1, y1)
plt.scatter(x2, y2, marker=~\textcolor{mtk25}{'v'}~, color=~\textcolor{mtk25}{'r'}~)
plt.scatter(x2, y3, marker=~\textcolor{mtk25}{'}~~\mtkCaret{mtk25}~~\textcolor{mtk25}{'}~, color=~\textcolor{mtk25}{'m'}~)
plt.title(~\textcolor{mtk25}{'Scatter Plot Example'}~)
plt.show()
\end{codeCell}
\end{paracol}

\columnratio{0.09}
\begin{paracol}{2}
\begin{cellExecute}[escapechar=~]
~\outputPrompt{4}~
\end{cellExecute}
\switchcolumn
\begin{resultCell}[escapechar=~]
\end{resultCell}
\end{paracol}

\begin{figure}[H]
\centering
\includegraphics[keepaspectratio,width=0.4\hsize]{/Users/watagashi0619/colaboratoryfigures/fig191125064527_10.png}
\end{figure}
\subsubsection{Stack Plots}

\columnratio{0.09}
\begin{paracol}{2}
\smallskip
\begin{cellExecute}[escapechar=~]
~\inputPrompt{5}~
\end{cellExecute}
\switchcolumn
\begin{codeCell}[escapechar=~]
~\textcolor{mtk17}{import}~ matplotlib.pyplot ~\textcolor{mtk17}{as}~ plt

idxes = [ ~\textcolor{mtk7}{1}~,  ~\textcolor{mtk7}{2}~,  ~\textcolor{mtk7}{3}~,  ~\textcolor{mtk7}{4}~,  ~\textcolor{mtk7}{5}~,  ~\textcolor{mtk7}{6}~,  ~\textcolor{mtk7}{7}~,  ~\textcolor{mtk7}{8}~,  ~\textcolor{mtk7}{9}~]
arr1  = [~\textcolor{mtk7}{23}~, ~\textcolor{mtk7}{40}~, ~\textcolor{mtk7}{28}~, ~\textcolor{mtk7}{43}~,  ~\textcolor{mtk7}{8}~, ~\textcolor{mtk7}{44}~, ~\textcolor{mtk7}{43}~, ~\textcolor{mtk7}{18}~, ~\textcolor{mtk7}{17}~]
arr2  = [~\textcolor{mtk7}{17}~, ~\textcolor{mtk7}{30}~, ~\textcolor{mtk7}{22}~, ~\textcolor{mtk7}{14}~, ~\textcolor{mtk7}{17}~, ~\textcolor{mtk7}{17}~, ~\textcolor{mtk7}{29}~, ~\textcolor{mtk7}{22}~, ~\textcolor{mtk7}{30}~]
arr3  = [~\textcolor{mtk7}{15}~, ~\textcolor{mtk7}{31}~, ~\textcolor{mtk7}{18}~, ~\textcolor{mtk7}{22}~, ~\textcolor{mtk7}{18}~, ~\textcolor{mtk7}{19}~, ~\textcolor{mtk7}{13}~, ~\textcolor{mtk7}{32}~, ~\textcolor{mtk7}{39}~]

~\mtkHash{mtk8}~~\textcolor{mtk8}{ Adding legend for stack plots is tricky.}~
plt.plot([], [], color=~\textcolor{mtk25}{'r'}~, label = ~\textcolor{mtk25}{'D 1'}~)
plt.plot([], [], color=~\textcolor{mtk25}{'g'}~, label = ~\textcolor{mtk25}{'D 2'}~)
plt.plot([], [], color=~\textcolor{mtk25}{'b'}~, label = ~\textcolor{mtk25}{'D 3'}~)

plt.stackplot(idxes, arr1, arr2, arr3, colors= [~\textcolor{mtk25}{'r'}~, ~\textcolor{mtk25}{'g'}~, ~\textcolor{mtk25}{'b'}~])
plt.title(~\textcolor{mtk25}{'Stack Plot Example'}~)
plt.legend()
plt.show()
\end{codeCell}
\end{paracol}

\columnratio{0.09}
\begin{paracol}{2}
\begin{cellExecute}[escapechar=~]
~\outputPrompt{5}~
\end{cellExecute}
\switchcolumn
\begin{resultCell}[escapechar=~]
\end{resultCell}
\end{paracol}

\begin{figure}[H]
\centering
\includegraphics[keepaspectratio,width=0.4\hsize]{/Users/watagashi0619/colaboratoryfigures/fig191125064527_12.png}
\end{figure}
\subsubsection{Pie Charts}

\columnratio{0.09}
\begin{paracol}{2}
\smallskip
\begin{cellExecute}[escapechar=~]
~\inputPrompt{6}~
\end{cellExecute}
\switchcolumn
\begin{codeCell}[escapechar=~]
~\textcolor{mtk17}{import}~ matplotlib.pyplot ~\textcolor{mtk17}{as}~ plt

labels = ~\textcolor{mtk25}{'S1'}~, ~\textcolor{mtk25}{'S2'}~, ~\textcolor{mtk25}{'S3'}~
sections = [~\textcolor{mtk7}{56}~, ~\textcolor{mtk7}{66}~, ~\textcolor{mtk7}{24}~]
colors = [~\textcolor{mtk25}{'c'}~, ~\textcolor{mtk25}{'g'}~, ~\textcolor{mtk25}{'y'}~]

plt.pie(sections, labels=labels, colors=colors,
        startangle=~\textcolor{mtk7}{90}~,
        explode = (~\textcolor{mtk7}{0}~, ~\textcolor{mtk7}{0.1}~, ~\textcolor{mtk7}{0}~),
        autopct = ~\textcolor{mtk25}{'}~~\mtkPercent{mtk25}~~\textcolor{mtk25}{1.2f}~~\mtkPercent{mtk25}~~\mtkPercent{mtk25}~~\textcolor{mtk25}{'}~)

plt.axis(~\textcolor{mtk25}{'equal'}~) ~\mtkHash{mtk8}~~\textcolor{mtk8}{ Try commenting this out.}~
plt.title(~\textcolor{mtk25}{'Pie Chart Example'}~)
plt.show()
\end{codeCell}
\end{paracol}

\columnratio{0.09}
\begin{paracol}{2}
\begin{cellExecute}[escapechar=~]
~\outputPrompt{6}~
\end{cellExecute}
\switchcolumn
\begin{resultCell}[escapechar=~]
\end{resultCell}
\end{paracol}

\begin{figure}[H]
\centering
\includegraphics[keepaspectratio,width=0.4\hsize]{/Users/watagashi0619/colaboratoryfigures/fig191125064527_14.png}
\end{figure}
\subsubsection{fill\_between and alpha}

\columnratio{0.09}
\begin{paracol}{2}
\smallskip
\begin{cellExecute}[escapechar=~]
~\inputPrompt{7}~
\end{cellExecute}
\switchcolumn
\begin{codeCell}[escapechar=~]
~\textcolor{mtk17}{import}~ matplotlib.pyplot ~\textcolor{mtk17}{as}~ plt
~\textcolor{mtk17}{import}~ numpy ~\textcolor{mtk17}{as}~ np

ys = ~\textcolor{mtk7}{200}~ + np.random.randn(~\textcolor{mtk7}{100}~)
x = [x ~\textcolor{mtk17}{for}~ x ~\textcolor{mtk6}{in}~ ~\textcolor{mtk13}{range}~(~\textcolor{mtk13}{len}~(ys))]

plt.plot(x, ys, ~\textcolor{mtk25}{'-'}~)
plt.fill_between(x, ys, ~\textcolor{mtk7}{195}~, where=(ys > ~\textcolor{mtk7}{195}~), facecolor=~\textcolor{mtk25}{'g'}~, alpha=~\textcolor{mtk7}{0.6}~)

plt.title(~\textcolor{mtk25}{"Fills and Alpha Example"}~)
plt.show()
\end{codeCell}
\end{paracol}

\columnratio{0.09}
\begin{paracol}{2}
\begin{cellExecute}[escapechar=~]
~\outputPrompt{7}~
\end{cellExecute}
\switchcolumn
\begin{resultCell}[escapechar=~]
\end{resultCell}
\end{paracol}

\begin{figure}[H]
\centering
\includegraphics[keepaspectratio,width=0.4\hsize]{/Users/watagashi0619/colaboratoryfigures/fig191125064527_16.png}
\end{figure}
\subsubsection{Subplotting using Subplot2grid}

\columnratio{0.09}
\begin{paracol}{2}
\smallskip
\begin{cellExecute}[escapechar=~]
~\inputPrompt{8}~
\end{cellExecute}
\switchcolumn
\begin{codeCell}[escapechar=~]
~\textcolor{mtk17}{import}~ matplotlib.pyplot ~\textcolor{mtk17}{as}~ plt
~\textcolor{mtk17}{import}~ numpy ~\textcolor{mtk17}{as}~ np

~\textcolor{mtk6}{def}~ ~\textcolor{mtk13}{random}~~\mtkTilde{mtk13}~~\textcolor{mtk13}{plots}~():
  xs = []
  ys = []
  
  ~\textcolor{mtk17}{for}~ i ~\textcolor{mtk6}{in}~ ~\textcolor{mtk13}{range}~(~\textcolor{mtk7}{20}~):
    x = i
    y = np.random.randint(~\textcolor{mtk7}{10}~)
    
    xs.append(x)
    ys.append(y)
  
  ~\textcolor{mtk17}{return}~ xs, ys

fig = plt.figure()
ax1 = plt.subplot2grid((~\textcolor{mtk7}{5}~, ~\textcolor{mtk7}{2}~), (~\textcolor{mtk7}{0}~, ~\textcolor{mtk7}{0}~), rowspan=~\textcolor{mtk7}{1}~, colspan=~\textcolor{mtk7}{2}~)
ax2 = plt.subplot2grid((~\textcolor{mtk7}{5}~, ~\textcolor{mtk7}{2}~), (~\textcolor{mtk7}{1}~, ~\textcolor{mtk7}{0}~), rowspan=~\textcolor{mtk7}{3}~, colspan=~\textcolor{mtk7}{2}~)
ax3 = plt.subplot2grid((~\textcolor{mtk7}{5}~, ~\textcolor{mtk7}{2}~), (~\textcolor{mtk7}{4}~, ~\textcolor{mtk7}{0}~), rowspan=~\textcolor{mtk7}{1}~, colspan=~\textcolor{mtk7}{1}~)
ax4 = plt.subplot2grid((~\textcolor{mtk7}{5}~, ~\textcolor{mtk7}{2}~), (~\textcolor{mtk7}{4}~, ~\textcolor{mtk7}{1}~), rowspan=~\textcolor{mtk7}{1}~, colspan=~\textcolor{mtk7}{1}~)

x, y = random_plots()
ax1.plot(x, y)

x, y = random_plots()
ax2.plot(x, y)

x, y = random_plots()
ax3.plot(x, y)

x, y = random_plots()
ax4.plot(x, y)

plt.tight_layout()
plt.show()
\end{codeCell}
\end{paracol}

\columnratio{0.09}
\begin{paracol}{2}
\begin{cellExecute}[escapechar=~]
~\outputPrompt{8}~
\end{cellExecute}
\switchcolumn
\begin{resultCell}[escapechar=~]
\end{resultCell}
\end{paracol}

\begin{figure}[H]
\centering
\includegraphics[keepaspectratio,width=0.4\hsize]{/Users/watagashi0619/colaboratoryfigures/fig191125064527_18.png}
\end{figure}
\subsection{Plot styles}
Colaboratory charts use \href{https://seaborn.pydata.org}{Seaborn's} custom styling by default. To customize styling further please see the \href{https://matplotlib.org/users/style\_sheets.html}{matplotlib docs}.\\
\subsection{3D Graphs}
\subsubsection{3D Scatter Plots}

\columnratio{0.09}
\begin{paracol}{2}
\smallskip
\begin{cellExecute}[escapechar=~]
~\inputPrompt{9}~
\end{cellExecute}
\switchcolumn
\begin{codeCell}[escapechar=~]
~\textcolor{mtk17}{import}~ matplotlib.pyplot ~\textcolor{mtk17}{as}~ plt
~\textcolor{mtk17}{import}~ numpy ~\textcolor{mtk17}{as}~ np
~\textcolor{mtk17}{from}~ mpl_toolkits.mplot3d ~\textcolor{mtk17}{import}~ axes3d

fig = plt.figure()
ax = fig.add_subplot(~\textcolor{mtk7}{111}~, projection = ~\textcolor{mtk25}{'3d'}~)

x1 = [~\textcolor{mtk7}{1}~, ~\textcolor{mtk7}{2}~, ~\textcolor{mtk7}{3}~, ~\textcolor{mtk7}{4}~, ~\textcolor{mtk7}{5}~, ~\textcolor{mtk7}{6}~, ~\textcolor{mtk7}{7}~, ~\textcolor{mtk7}{8}~, ~\textcolor{mtk7}{9}~, ~\textcolor{mtk7}{10}~]
y1 = np.random.randint(~\textcolor{mtk7}{10}~, size=~\textcolor{mtk7}{10}~)
z1 = np.random.randint(~\textcolor{mtk7}{10}~, size=~\textcolor{mtk7}{10}~)

x2 = [~\textcolor{mtk7}{-1}~, ~\textcolor{mtk7}{-2}~, ~\textcolor{mtk7}{-3}~, ~\textcolor{mtk7}{-4}~, ~\textcolor{mtk7}{-5}~, ~\textcolor{mtk7}{-6}~, ~\textcolor{mtk7}{-7}~, ~\textcolor{mtk7}{-8}~, ~\textcolor{mtk7}{-9}~, ~\textcolor{mtk7}{-10}~]
y2 = np.random.randint(~\textcolor{mtk7}{-10}~, ~\textcolor{mtk7}{0}~, size=~\textcolor{mtk7}{10}~)
z2 = np.random.randint(~\textcolor{mtk7}{10}~, size=~\textcolor{mtk7}{10}~)

ax.scatter(x1, y1, z1, c=~\textcolor{mtk25}{'b'}~, marker=~\textcolor{mtk25}{'o'}~, label=~\textcolor{mtk25}{'blue'}~)
ax.scatter(x2, y2, z2, c=~\textcolor{mtk25}{'g'}~, marker=~\textcolor{mtk25}{'D'}~, label=~\textcolor{mtk25}{'green'}~)

ax.set_xlabel(~\textcolor{mtk25}{'x axis'}~)
ax.set_ylabel(~\textcolor{mtk25}{'y axis'}~)
ax.set_zlabel(~\textcolor{mtk25}{'z axis'}~)
plt.title(~\textcolor{mtk25}{"3D Scatter Plot Example"}~)
plt.legend()
plt.tight_layout()
plt.show()
\end{codeCell}
\end{paracol}

\columnratio{0.09}
\begin{paracol}{2}
\begin{cellExecute}[escapechar=~]
~\outputPrompt{9}~
\end{cellExecute}
\switchcolumn
\begin{resultCell}[escapechar=~]
\end{resultCell}
\end{paracol}

\begin{figure}[H]
\centering
\includegraphics[keepaspectratio,width=0.4\hsize]{/Users/watagashi0619/colaboratoryfigures/fig191125064527_22.png}
\end{figure}
\subsubsection{3D Bar Plots}

\columnratio{0.09}
\begin{paracol}{2}
\smallskip
\begin{cellExecute}[escapechar=~]
~\inputPrompt{10}~
\end{cellExecute}
\switchcolumn
\begin{codeCell}[escapechar=~]
~\textcolor{mtk17}{import}~ matplotlib.pyplot ~\textcolor{mtk17}{as}~ plt
~\textcolor{mtk17}{import}~ numpy ~\textcolor{mtk17}{as}~ np

fig = plt.figure()
ax = fig.add_subplot(~\textcolor{mtk7}{111}~, projection = ~\textcolor{mtk25}{'3d'}~)

x = [~\textcolor{mtk7}{1}~, ~\textcolor{mtk7}{2}~, ~\textcolor{mtk7}{3}~, ~\textcolor{mtk7}{4}~, ~\textcolor{mtk7}{5}~, ~\textcolor{mtk7}{6}~, ~\textcolor{mtk7}{7}~, ~\textcolor{mtk7}{8}~, ~\textcolor{mtk7}{9}~, ~\textcolor{mtk7}{10}~]
y = np.random.randint(~\textcolor{mtk7}{10}~, size=~\textcolor{mtk7}{10}~)
z = np.zeros(~\textcolor{mtk7}{10}~)

dx = np.ones(~\textcolor{mtk7}{10}~)
dy = np.ones(~\textcolor{mtk7}{10}~)
dz = [~\textcolor{mtk7}{1}~, ~\textcolor{mtk7}{2}~, ~\textcolor{mtk7}{3}~, ~\textcolor{mtk7}{4}~, ~\textcolor{mtk7}{5}~, ~\textcolor{mtk7}{6}~, ~\textcolor{mtk7}{7}~, ~\textcolor{mtk7}{8}~, ~\textcolor{mtk7}{9}~, ~\textcolor{mtk7}{10}~]

ax.bar3d(x, y, z, dx, dy, dz, color=~\textcolor{mtk25}{'g'}~)

ax.set_xlabel(~\textcolor{mtk25}{'x axis'}~)
ax.set_ylabel(~\textcolor{mtk25}{'y axis'}~)
ax.set_zlabel(~\textcolor{mtk25}{'z axis'}~)
plt.title(~\textcolor{mtk25}{"3D Bar Chart Example"}~)
plt.tight_layout()
plt.show()
\end{codeCell}
\end{paracol}

\columnratio{0.09}
\begin{paracol}{2}
\begin{cellExecute}[escapechar=~]
~\outputPrompt{10}~
\end{cellExecute}
\switchcolumn
\begin{resultCell}[escapechar=~]
\end{resultCell}
\end{paracol}

\begin{figure}[H]
\centering
\includegraphics[keepaspectratio,width=0.4\hsize]{/Users/watagashi0619/colaboratoryfigures/fig191125064527_24.png}
\end{figure}
\subsubsection{Wireframe Plots}

\columnratio{0.09}
\begin{paracol}{2}
\smallskip
\begin{cellExecute}[escapechar=~]
~\inputPrompt{11}~
\end{cellExecute}
\switchcolumn
\begin{codeCell}[escapechar=~]
~\textcolor{mtk17}{import}~ matplotlib.pyplot ~\textcolor{mtk17}{as}~ plt

fig = plt.figure()
ax = fig.add_subplot(~\textcolor{mtk7}{111}~, projection = ~\textcolor{mtk25}{'3d'}~)

x, y, z = axes3d.get_test_data()

ax.plot_wireframe(x, y, z, rstride = ~\textcolor{mtk7}{2}~, cstride = ~\textcolor{mtk7}{2}~)

plt.title(~\textcolor{mtk25}{"Wireframe Plot Example"}~)
plt.tight_layout()
plt.show()
\end{codeCell}
\end{paracol}

\columnratio{0.09}
\begin{paracol}{2}
\begin{cellExecute}[escapechar=~]
~\outputPrompt{11}~
\end{cellExecute}
\switchcolumn
\begin{resultCell}[escapechar=~]
\end{resultCell}
\end{paracol}

\begin{figure}[H]
\centering
\includegraphics[keepaspectratio,width=0.4\hsize]{/Users/watagashi0619/colaboratoryfigures/fig191125064527_26.png}
\end{figure}
\subsection{Seaborn}
There are several libraries layered on top of Matplotlib that you can use in Colab. One that is worth highlighting is \href{http://seaborn.pydata.org}{Seaborn}:\\

\columnratio{0.09}
\begin{paracol}{2}
\smallskip
\begin{cellExecute}[escapechar=~]
~\inputPrompt{12}~
\end{cellExecute}
\switchcolumn
\begin{codeCell}[escapechar=~]
~\textcolor{mtk17}{import}~ matplotlib.pyplot ~\textcolor{mtk17}{as}~ plt
~\textcolor{mtk17}{import}~ numpy ~\textcolor{mtk17}{as}~ np
~\textcolor{mtk17}{import}~ seaborn ~\textcolor{mtk17}{as}~ sns

~\mtkHash{mtk8}~~\textcolor{mtk8}{ Generate some random data}~
num_points = ~\textcolor{mtk7}{20}~
~\mtkHash{mtk8}~~\textcolor{mtk8}{ x will be 5, 6, 7... but also twiddled randomly}~
x = ~\textcolor{mtk7}{5}~ + np.arange(num_points) + np.random.randn(num_points)
~\mtkHash{mtk8}~~\textcolor{mtk8}{ y will be 10, 11, 12... but twiddled even more r}~~\textcolor{mtk8}{andomly}~
y = ~\textcolor{mtk7}{10}~ + np.arange(num_points) + ~\textcolor{mtk7}{5}~ * np.random.randn(num_points)
sns.regplot(x, y)
plt.show()
\end{codeCell}
\end{paracol}

\columnratio{0.09}
\begin{paracol}{2}
\begin{cellExecute}[escapechar=~]
~\outputPrompt{12}~
\end{cellExecute}
\switchcolumn
\begin{resultCell}[escapechar=~]
\end{resultCell}
\end{paracol}

\begin{figure}[H]
\centering
\includegraphics[keepaspectratio,width=0.4\hsize]{/Users/watagashi0619/colaboratoryfigures/fig191125064527_28.png}
\end{figure}
That's a simple scatterplot with a nice regression line fit to it, all with just one call to Seaborn's \href{http://seaborn.pydata.org/generated/seaborn.regplot.html#seaborn.regplot}{regplot}.\\
Here's a Seaborn \href{https://seaborn.pydata.org/generated/seaborn.heatmap.html}{heatmap}:\\

\columnratio{0.09}
\begin{paracol}{2}
\smallskip
\begin{cellExecute}[escapechar=~]
~\inputPrompt{13}~
\end{cellExecute}
\switchcolumn
\begin{codeCell}[escapechar=~]
~\textcolor{mtk17}{import}~ matplotlib.pyplot ~\textcolor{mtk17}{as}~ plt
~\textcolor{mtk17}{import}~ numpy ~\textcolor{mtk17}{as}~ np

~\mtkHash{mtk8}~~\textcolor{mtk8}{ Make a 10 x 10 heatmap of some random data}~
side_length = ~\textcolor{mtk7}{10}~
~\mtkHash{mtk8}~~\textcolor{mtk8}{ Start with a 10 x 10 matrix with values randomiz}~~\textcolor{mtk8}{ed around 5}~
data = ~\textcolor{mtk7}{5}~ + np.random.randn(side_length, side_length)
~\mtkHash{mtk8}~~\textcolor{mtk8}{ The next two lines make the values larger as we }~~\textcolor{mtk8}{get closer to (9, 9)}~
data += np.arange(side_length)
data += np.reshape(np.arange(side_length), (side_length, ~\textcolor{mtk7}{1}~))
~\mtkHash{mtk8}~~\textcolor{mtk8}{ Generate the heatmap}~
sns.heatmap(data)
plt.show()
\end{codeCell}
\end{paracol}

\columnratio{0.09}
\begin{paracol}{2}
\begin{cellExecute}[escapechar=~]
~\outputPrompt{13}~
\end{cellExecute}
\switchcolumn
\begin{resultCell}[escapechar=~]
\end{resultCell}
\end{paracol}

\begin{figure}[H]
\centering
\includegraphics[keepaspectratio,width=0.4\hsize]{/Users/watagashi0619/colaboratoryfigures/fig191125064527_30.png}
\end{figure}
\subsection{Altair}
\href{http://altair-viz.github.io}{Altair} is a declarative visualization library for creating interactive visualizations in Python, and is installed and enabled in Colab by default.\\
For example, here is an interactive scatter plot:\\

\columnratio{0.09}
\begin{paracol}{2}
\smallskip
\begin{cellExecute}[escapechar=~]
~\inputPrompt{14}~
\end{cellExecute}
\switchcolumn
\begin{codeCell}[escapechar=~]
~\textcolor{mtk17}{import}~ altair ~\textcolor{mtk17}{as}~ alt
~\textcolor{mtk17}{from}~ vega_datasets ~\textcolor{mtk17}{import}~ data
cars = data.cars()

alt.Chart(cars).mark_point().encode(
    x=~\textcolor{mtk25}{'Horsepower'}~,
    y=~\textcolor{mtk25}{'Miles}~~\mtkTilde{mtk25}~~\textcolor{mtk25}{per}~~\mtkTilde{mtk25}~~\textcolor{mtk25}{Gallon'}~,
    color=~\textcolor{mtk25}{'Origin'}~,
).interactive()
\end{codeCell}
\end{paracol}

\columnratio{0.09}
\begin{paracol}{2}
\begin{cellExecute}[escapechar=~]
~\outputPrompt{14}~
\end{cellExecute}
\switchcolumn
\begin{resultCell}[escapechar=~]
\end{resultCell}
\end{paracol}
For more examples of Altair plots, see the \href{/notebooks/snippets/altair.ipynb}{Altair snippets notebook} or the external \href{https://altair-viz.github.io/gallery/}{Altair Example Gallery}.\\
\subsection{Plotly}
\subsubsection{Cell configuration}
This method pre-populates the outputframe with the configuration that Plotly expects and must be executed for every cell which is displaying a Plotly graph.\\

\columnratio{0.09}
\begin{paracol}{2}
\smallskip
\begin{cellExecute}[escapechar=~]
~\inputPrompt{15}~
\end{cellExecute}
\switchcolumn
\begin{codeCell}[escapechar=~]
~\textcolor{mtk6}{def}~ ~\textcolor{mtk13}{enable}~~\mtkTilde{mtk13}~~\textcolor{mtk13}{plotly}~~\mtkTilde{mtk13}~~\textcolor{mtk13}{in}~~\mtkTilde{mtk13}~~\textcolor{mtk13}{cell}~():
  ~\textcolor{mtk17}{import}~ IPython
  ~\textcolor{mtk17}{from}~ plotly.offline ~\textcolor{mtk17}{import}~ init_notebook_mode
  display(IPython.core.display.HTML(~\textcolor{mtk25}{'''}~
        ~\textcolor{mtk11}{<}~~\textcolor{mtk19}{script}~ ~\textcolor{mtk4}{src}~~\textcolor{mtk11}{=}~~\textcolor{mtk6}{"/static/components/requirejs/require.js"}~~\textcolor{mtk11}{></}~~\textcolor{mtk19}{script}~~\textcolor{mtk11}{>}~
  ~\textcolor{mtk25}{'''}~))
  init_notebook_mode(connected=~\textcolor{mtk6}{False}~)
\end{codeCell}
\end{paracol}
\subsubsection{Sample}

\columnratio{0.09}
\begin{paracol}{2}
\smallskip
\begin{cellExecute}[escapechar=~]
~\inputPrompt{16}~
\end{cellExecute}
\switchcolumn
\begin{codeCell}[escapechar=~]
~\textcolor{mtk17}{from}~ plotly.offline ~\textcolor{mtk17}{import}~ iplot
~\textcolor{mtk17}{import}~ plotly.graph_objs ~\textcolor{mtk17}{as}~ go

enable_plotly_in_cell()

data = [
    go.Contour(
        z=[[~\textcolor{mtk7}{10}~, ~\textcolor{mtk7}{10.625}~, ~\textcolor{mtk7}{12.5}~, ~\textcolor{mtk7}{15.625}~, ~\textcolor{mtk7}{20}~],
           [~\textcolor{mtk7}{5.625}~, ~\textcolor{mtk7}{6.25}~, ~\textcolor{mtk7}{8.125}~, ~\textcolor{mtk7}{11.25}~, ~\textcolor{mtk7}{15.625}~],
           [~\textcolor{mtk7}{2.5}~, ~\textcolor{mtk7}{3.125}~, ~\textcolor{mtk7}{5}~., ~\textcolor{mtk7}{8.125}~, ~\textcolor{mtk7}{12.5}~],
           [~\textcolor{mtk7}{0.625}~, ~\textcolor{mtk7}{1.25}~, ~\textcolor{mtk7}{3.125}~, ~\textcolor{mtk7}{6.25}~, ~\textcolor{mtk7}{10.625}~],
           [~\textcolor{mtk7}{0}~, ~\textcolor{mtk7}{0.625}~, ~\textcolor{mtk7}{2.5}~, ~\textcolor{mtk7}{5.625}~, ~\textcolor{mtk7}{10}~]]
    )
]
iplot(data)
\end{codeCell}
\end{paracol}

\columnratio{0.09}
\begin{paracol}{2}
\begin{cellExecute}[escapechar=~]
~\outputPrompt{16}~
\end{cellExecute}
\switchcolumn
\begin{resultCell}[escapechar=~]
\end{resultCell}
\end{paracol}
\subsubsection{Plotly Pre-execute Hook}
If you wish to automatically load the required resources within each cell, you can add the enable\_plotly\_in\_cell function to a Jupyter pre-execute hook and it will be automaticaly executed before any cell execution:\\

\columnratio{0.09}
\begin{paracol}{2}
\smallskip
\begin{cellExecute}[escapechar=~]
~\inputPrompt{17}~
\end{cellExecute}
\switchcolumn
\begin{codeCell}[escapechar=~]
get_ipython().events.register(~\textcolor{mtk25}{'pre}~~\mtkTilde{mtk25}~~\textcolor{mtk25}{run}~~\mtkTilde{mtk25}~~\textcolor{mtk25}{cell'}~, enable_plotly_in_cell)
\end{codeCell}
\end{paracol}
Because this pre-run hook causes additional javascript resources to be loaded in each cell output, we will disable it here:\\

\columnratio{0.09}
\begin{paracol}{2}
\smallskip
\begin{cellExecute}[escapechar=~]
~\inputPrompt{18}~
\end{cellExecute}
\switchcolumn
\begin{codeCell}[escapechar=~]
get_ipython().events.unregister(~\textcolor{mtk25}{'pre}~~\mtkTilde{mtk25}~~\textcolor{mtk25}{run}~~\mtkTilde{mtk25}~~\textcolor{mtk25}{cell'}~, enable_plotly_in_cell)
\end{codeCell}
\end{paracol}

\columnratio{0.09}
\begin{paracol}{2}
\begin{cellExecute}[escapechar=~]
~\outputPrompt{18}~
\end{cellExecute}
\switchcolumn
\begin{resultCell}[escapechar=~]
\end{resultCell}
\end{paracol}
\subsection{Bokeh}
\subsubsection{Sample}

\columnratio{0.09}
\begin{paracol}{2}
\smallskip
\begin{cellExecute}[escapechar=~]
~\inputPrompt{19}~
\end{cellExecute}
\switchcolumn
\begin{codeCell}[escapechar=~]
~\textcolor{mtk17}{import}~ numpy ~\textcolor{mtk17}{as}~ np
~\textcolor{mtk17}{from}~ bokeh.plotting ~\textcolor{mtk17}{import}~ figure, show
~\textcolor{mtk17}{from}~ bokeh.io ~\textcolor{mtk17}{import}~ output_notebook

~\mtkHash{mtk8}~~\textcolor{mtk8}{ Call once to configure Bokeh to display plots in}~~\textcolor{mtk8}{line in the notebook.}~
output_notebook()
\end{codeCell}
\end{paracol}

\columnratio{0.09}
\begin{paracol}{2}
\smallskip
\begin{cellExecute}[escapechar=~]
~\inputPrompt{20}~
\end{cellExecute}
\switchcolumn
\begin{codeCell}[escapechar=~]
N = ~\textcolor{mtk7}{4000}~
x = np.random.random(size=N) * ~\textcolor{mtk7}{100}~
y = np.random.random(size=N) * ~\textcolor{mtk7}{100}~
radii = np.random.random(size=N) * ~\textcolor{mtk7}{1.5}~
colors = [~\textcolor{mtk25}{"}~~\mtkHash{mtk25}~~\mtkPercent{mtk25}~~\textcolor{mtk25}{02x}~~\mtkPercent{mtk25}~~\textcolor{mtk25}{02x}~~\mtkPercent{mtk25}~~\textcolor{mtk25}{02x"}~ % (r, g, ~\textcolor{mtk7}{150}~) ~\textcolor{mtk17}{for}~ r, g ~\textcolor{mtk6}{in}~ ~\textcolor{mtk13}{zip}~(np.floor(~\textcolor{mtk7}{50}~+~\textcolor{mtk7}{2}~*x).astype(~\textcolor{mtk12}{int}~), np.floor(~\textcolor{mtk7}{30}~+~\textcolor{mtk7}{2}~*y).astype(~\textcolor{mtk12}{int}~))]

p = figure()
p.circle(x, y, radius=radii, fill_color=colors, fill_alpha=~\textcolor{mtk7}{0.6}~, line_color=~\textcolor{mtk6}{None}~)
show(p)
\end{codeCell}
\end{paracol}

\columnratio{0.09}
\begin{paracol}{2}
\begin{cellExecute}[escapechar=~]
~\outputPrompt{20}~
\end{cellExecute}
\switchcolumn
\begin{resultCell}[escapechar=~]
\end{resultCell}
\end{paracol}
