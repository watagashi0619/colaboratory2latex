\section{はじめにこのパッケージの簡単な説明}
\par 思いつくままに書いているので、まとまりがないですがお許しください。主にmarkdownの対応について書いています。
\par 最初のセルの部分にコメントをつけてみる。例えば今回の場合、
\begin{markdownCodeCell}[escapechar=~]
# 27;figureoption:keepaspectratio,width=0.9\hsize
# 32;figureoption:keepaspectratio,width=0.75\hsize

\end{markdownCodeCell}\par と書いています。「\#セル番号;figureoption:オプション」とすることで、一番はじめのセルを1番として、指定番目のセルで表示されているtexのfigureの
\begin{markdownCodeCell}[escapechar=~]
\includegraphics[ここ!]{なんか.png}

\end{markdownCodeCell}\par ここ!の部分の設定を指定できます。
また、このようにコードをmarkdownの方に貼り付けも使えます。
\par また、引用もできます。
\begin{quote}

texのquoteに対応させています。

\end{quote}
\par また、このpdfのもととなったgoogle colaboratoryのリンクは\href{https://colab.research.google.com/drive/1ciNys02sU0IC8EuUC5A0\_y13wfQda-7k}{これ}なのですが、このようにリンクにも対応しています。また、
\begin{itemize}

\item itemize
\item enumerate

\end{itemize}
\par については、このような一段階までなら対応していますが、二段以上の入れ子は対応していません。
また、\textbf{強調}、 \textit{italic(ただし英語のみ)}にも対応しています。

\hrulefill

\par 線も一応引けます。
\par シャープ は1つでsection
\subsection{2つでsubsection}
\subsubsection{3つでsubsubsection}
\paragraph{4つでpragraph}
\subparagraph{5つでsubparagraphに対応します。}
\par また、このように
\begin{table}[H]

\centering\footnotesize
\begin{tabular}{|l|l|l|}
\hline
分布名 & ポアソン分布 & ガウス分布 \\ \hline\hline
確率密度関数 & $\frac{\mu^x}{x!}e^{-\mu}$ & $\frac{1}{\sqrt{2\pi\sigma^2}}\exp\left(-\frac{(x-\mu)^2}{2\sigma^2}\right)$ \\ \hline
平均値 & $\mu$ & $\mu$ \\ \hline
分散 & $\mu$ & $\sigma^2$ \\ \hline
--- & --- & --- \\ \hline
\end{tabular}
\end{table}

\par テーブルにも対応しています。pythonコード実行によって作成されたテーブルも、うまく出力されるはずです(iminuitのフィッティングによるものはしっかりと表示できました。)
さらに、数式に対応していることもわかると思います。こういうタイプのものもいけるということで、とりあえずオイラーの等式をだしてみます。
\[
e^{i\pi}+1=0
\]
ただし、equationを使おうとすると、エラーを吐く場合があるので、うまくあとで調整してください。
\par 以下は、実際のレポートのようなものです。雰囲気をなんとなくでも掴めるのではないでしょうか。

\hrulefill

\par ※このレポートはデモのために適当に作成したものです。変数名の雑さなどには目を瞑ってください。ただし、取ったデータなどは実際のものです。
\section{概要}
\par 以前のドミノピザのRTキャンペーンで、当たりの数や出る時間に明らかな偏りがあるように思えたため、軽く統計をとって調べてることにした。
\section{データの取得方法}
\par 別途用意したpythonプログラム(後述)で適当にtwitterのapi叩いてドミノピザ公式が当たり、はずれ、という画像とともにリプライを飛ばしてるツイートを取得して集計。当たりなら1、はずれなら0をtxtファイルにタイムスタンプと一緒に記録。cronによる定期実行、2分毎に直近2分間最大3200件のツイートの結果が各々のtxtファイルに記録されるようにした。
\subsection{txtファイル作成に使用したプログラム}
\par twitterからのテータの収集にはtweepyを使用しています。
\begin{markdownCodeCell}[escapechar=~]
import tweepy
import datetime
import re
 
 
CK="Consumer Key"
CS="Consumer Secret"
AT="Access Token"
AS="Access Token Secret"
 
 
auth = tweepy.OAuthHandler(CK, CS)
auth.set_access_token(AT, AS)
api = tweepy.API(auth)
 
 
dt_now = datetime.datetime.now()
dt_L = datetime.datetime(*dt_now.timetuple()[:5]) - datetime.timedelta(minutes=2)
dt_R = datetime.datetime(*dt_now.timetuple()[:5])
filename = dt_now.strftime('\%y\%m\%d\%H\%M\%S') + '.txt'
 
 
with open(filename,'w') as f:
    for i in range(1,17):
        res = api.user_timeline(screen_name="dominos_JP", count=200, page=i)
 
 
        for tweet in res:
            timestamp = datetime.datetime.fromtimestamp(((tweet.id >> 22)+1288834974657)/1000.0)
            if dt_L <= timestamp:
                if timestamp < dt_R:
                    if re.search(r'残念',tweet.text):
                        f.write(str(timestamp).split()[1] + ' 0' + '\n')
                    elif re.search(r'当選',tweet.text):
                        f.write(str(timestamp).split()[1] + ' 1' + '\n')
            else:
                break
        else:
            continue
        break
 
 

\end{markdownCodeCell}\section{2019年10月30日の集計結果}
\par この日は \href{https://twitter.com/dominos\_JP/status/1188213772197990402?ref\_src=twsrc\%5Etfw}{このキャンペーン} の最終日でした。
\subsection{下準備}
\par 分析にあたり、モジュールの読み込みを行っています。

\columnratio{0.09}
\begin{paracol}{2}
\smallskip
\begin{cellExecute}[escapechar=~]
~\inputPrompt{1}~
\end{cellExecute}
\switchcolumn
\begin{codeCell}[escapechar=~]
~\textcolor{mtk17}{import}~ os 
~\textcolor{mtk17}{import}~ pandas ~\textcolor{mtk17}{as}~ pd
~\textcolor{mtk17}{import}~ numpy ~\textcolor{mtk17}{as}~ np
~\textcolor{mtk17}{from}~ scipy ~\textcolor{mtk17}{import}~ interpolate 
~\textcolor{mtk17}{import}~ matplotlib.pyplot ~\textcolor{mtk17}{as}~ plt
~\textcolor{mtk17}{from}~ matplotlib ~\textcolor{mtk17}{import}~ dates ~\textcolor{mtk17}{as}~ mdates
~\textcolor{mtk17}{from}~ datetime ~\textcolor{mtk17}{import}~ datetime ~\textcolor{mtk17}{as}~ dt
\end{codeCell}
\end{paracol}
\par google colaboratory上のディレクトリにあらかじめ全txtファイルが存在しています。

\columnratio{0.09}
\begin{paracol}{2}
\smallskip
\begin{cellExecute}[escapechar=~]
~\inputPrompt{2}~
\end{cellExecute}
\switchcolumn
\begin{codeCell}[escapechar=~]
files=os.listdir(~\textcolor{mtk25}{'/content'}~)
\end{codeCell}
\end{paracol}

\columnratio{0.09}
\begin{paracol}{2}
\smallskip
\begin{cellExecute}[escapechar=~]
~\inputPrompt{3}~
\end{cellExecute}
\switchcolumn
\begin{codeCell}[escapechar=~]
files.remove(~\textcolor{mtk25}{'sample}~~\mtkTilde{mtk25}~~\textcolor{mtk25}{data'}~)
files.remove(~\textcolor{mtk25}{'.config'}~)
\end{codeCell}
\end{paracol}
\par 適当にデータを作成します。

\columnratio{0.09}
\begin{paracol}{2}
\smallskip
\begin{cellExecute}[escapechar=~]
~\inputPrompt{4}~
\end{cellExecute}
\switchcolumn
\begin{codeCell}[escapechar=~]
data=[]
~\textcolor{mtk17}{for}~ ~\textcolor{mtk14}{file}~ ~\textcolor{mtk6}{in}~ files:
  ~\textcolor{mtk17}{with}~ ~\textcolor{mtk13}{open}~(~\textcolor{mtk25}{'/content/'}~+~\textcolor{mtk14}{file}~, encoding=~\textcolor{mtk25}{"utf-8"}~) ~\textcolor{mtk17}{as}~ f:
    ~\textcolor{mtk17}{for}~ line ~\textcolor{mtk6}{in}~ f:
      line.strip(~\textcolor{mtk25}{'}~~\mtkBackslash{mtk25}~~\textcolor{mtk25}{n'}~)
      data.append(line.split())
\end{codeCell}
\end{paracol}

\columnratio{0.09}
\begin{paracol}{2}
\smallskip
\begin{cellExecute}[escapechar=~]
~\inputPrompt{5}~
\end{cellExecute}
\switchcolumn
\begin{codeCell}[escapechar=~]
data=[[dt.strptime(~\textcolor{mtk25}{'2019-10-29 '}~+[datum[~\textcolor{mtk7}{0}~][:~\textcolor{mtk7}{-3}~],datum[~\textcolor{mtk7}{0}~]+~\textcolor{mtk25}{'.000'}~][~\textcolor{mtk13}{len}~(datum[~\textcolor{mtk7}{0}~])<~\textcolor{mtk7}{9}~], ~\textcolor{mtk25}{'}~~\mtkPercent{mtk25}~~\textcolor{mtk25}{Y-}~~\mtkPercent{mtk25}~~\textcolor{mtk25}{m-}~~\mtkPercent{mtk25}~~\textcolor{mtk25}{d }~~\mtkPercent{mtk25}~~\textcolor{mtk25}{H:}~~\mtkPercent{mtk25}~~\textcolor{mtk25}{M:}~~\mtkPercent{mtk25}~~\textcolor{mtk25}{S.}~~\mtkPercent{mtk25}~~\textcolor{mtk25}{f'}~),~\textcolor{mtk12}{int}~(datum[~\textcolor{mtk7}{1}~])] ~\textcolor{mtk17}{for}~ datum ~\textcolor{mtk6}{in}~ data]
data=~\textcolor{mtk13}{sorted}~(data)
\end{codeCell}
\end{paracol}

\columnratio{0.09}
\begin{paracol}{2}
\smallskip
\begin{cellExecute}[escapechar=~]
~\inputPrompt{6}~
\end{cellExecute}
\switchcolumn
\begin{codeCell}[escapechar=~]
x=[a[~\textcolor{mtk7}{0}~] ~\textcolor{mtk17}{for}~ a ~\textcolor{mtk6}{in}~ data]
\end{codeCell}
\end{paracol}

\columnratio{0.09}
\begin{paracol}{2}
\smallskip
\begin{cellExecute}[escapechar=~]
~\inputPrompt{7}~
\end{cellExecute}
\switchcolumn
\begin{codeCell}[escapechar=~]
y=[a[~\textcolor{mtk7}{1}~] ~\textcolor{mtk17}{for}~ a ~\textcolor{mtk6}{in}~ data]
\end{codeCell}
\end{paracol}

\columnratio{0.09}
\begin{paracol}{2}
\smallskip
\begin{cellExecute}[escapechar=~]
~\inputPrompt{8}~
\end{cellExecute}
\switchcolumn
\begin{codeCell}[escapechar=~]
lll=~\textcolor{mtk7}{0}~
rrr=~\textcolor{mtk7}{0}~
~\textcolor{mtk17}{for}~ i ~\textcolor{mtk6}{in}~ ~\textcolor{mtk13}{range}~(~\textcolor{mtk13}{len}~(data)):
  ~\textcolor{mtk17}{if}~ data[i][~\textcolor{mtk7}{0}~]>dt.strptime(~\textcolor{mtk25}{'2019-10-29 17:00'}~,~\textcolor{mtk25}{'}~~\mtkPercent{mtk25}~~\textcolor{mtk25}{Y-}~~\mtkPercent{mtk25}~~\textcolor{mtk25}{m-}~~\mtkPercent{mtk25}~~\textcolor{mtk25}{d }~~\mtkPercent{mtk25}~~\textcolor{mtk25}{H:}~~\mtkPercent{mtk25}~~\textcolor{mtk25}{M'}~):
    lll=i
    ~\textcolor{mtk17}{break}~
\end{codeCell}
\end{paracol}
\par 取得した全データ数は以下のようになっています。

\columnratio{0.09}
\begin{paracol}{2}
\smallskip
\begin{cellExecute}[escapechar=~]
~\inputPrompt{9}~
\end{cellExecute}
\switchcolumn
\begin{codeCell}[escapechar=~]
~\textcolor{mtk13}{len}~(data)
\end{codeCell}
\end{paracol}

\columnratio{0.09}
\begin{paracol}{2}
\begin{cellExecute}[escapechar=~]
~\outputPrompt{9}~
\end{cellExecute}
\switchcolumn
\begin{resultCell}[escapechar=~]
48259\end{resultCell}
\end{paracol}
\subsection{時間vs当たりはずれのプロット}
\par 当たりを1、はずれを0に、横を時間軸としてプロットしたものです。明らかに後半は当たりが出ていません。

\columnratio{0.09}
\begin{paracol}{2}
\smallskip
\begin{cellExecute}[escapechar=~]
~\inputPrompt{10}~
\end{cellExecute}
\switchcolumn
\begin{codeCell}[escapechar=~]
countwin=y_win_sum[~\textcolor{mtk7}{-1}~]-y_win_sum[lll]
countlose=y_lose_sum[~\textcolor{mtk7}{-1}~]-y_lose_sum[lll]
\end{codeCell}
\end{paracol}

\columnratio{0.09}
\begin{paracol}{2}
\smallskip
\begin{cellExecute}[escapechar=~]
~\inputPrompt{11}~
\end{cellExecute}
\switchcolumn
\begin{codeCell}[escapechar=~]
fig, ax = plt.subplots(figsize=(~\textcolor{mtk7}{18}~,~\textcolor{mtk7}{3}~))

ax.scatter(x,y,s=~\textcolor{mtk7}{1}~)

ax.xaxis.set_major_locator(mdates.HourLocator(interval=~\textcolor{mtk7}{2}~))
ax.xaxis.set_major_formatter(mdates.DateFormatter(~\textcolor{mtk25}{"}~~\mtkPercent{mtk25}~~\textcolor{mtk25}{Y-}~~\mtkPercent{mtk25}~~\textcolor{mtk25}{m-}~~\mtkPercent{mtk25}~~\textcolor{mtk25}{d }~~\mtkPercent{mtk25}~~\textcolor{mtk25}{H:}~~\mtkPercent{mtk25}~~\textcolor{mtk25}{M"}~))
ax.set_xlim([dt.strptime(~\textcolor{mtk25}{'2019-10-29 7:00'}~,~\textcolor{mtk25}{'}~~\mtkPercent{mtk25}~~\textcolor{mtk25}{Y-}~~\mtkPercent{mtk25}~~\textcolor{mtk25}{m-}~~\mtkPercent{mtk25}~~\textcolor{mtk25}{d }~~\mtkPercent{mtk25}~~\textcolor{mtk25}{H:}~~\mtkPercent{mtk25}~~\textcolor{mtk25}{M'}~), dt.strptime(~\textcolor{mtk25}{'2019-10-30 00:00'}~,~\textcolor{mtk25}{'}~~\mtkPercent{mtk25}~~\textcolor{mtk25}{Y-}~~\mtkPercent{mtk25}~~\textcolor{mtk25}{m-}~~\mtkPercent{mtk25}~~\textcolor{mtk25}{d }~~\mtkPercent{mtk25}~~\textcolor{mtk25}{H:}~~\mtkPercent{mtk25}~~\textcolor{mtk25}{M'}~)])

ax.set_ylabel(~\textcolor{mtk25}{"win(1) or lose(0)"}~)
ax.set_xlabel(~\textcolor{mtk25}{"Time"}~)

~\mtkHash{mtk8}~~\textcolor{mtk8}{ax.annotate("Win:}~~\mtkPercent{mtk8}~~\textcolor{mtk8}{d" }~~\mtkPercent{mtk8}~~\textcolor{mtk8}{ countwin, xy = (dt.strptim}~~\textcolor{mtk8}{e('2019-10-29 17:30','}~~\mtkPercent{mtk8}~~\textcolor{mtk8}{Y-}~~\mtkPercent{mtk8}~~\textcolor{mtk8}{m-}~~\mtkPercent{mtk8}~~\textcolor{mtk8}{d }~~\mtkPercent{mtk8}~~\textcolor{mtk8}{H:}~~\mtkPercent{mtk8}~~\textcolor{mtk8}{M'), 1), size =}~~\textcolor{mtk8}{ 10, color='r')}~
~\mtkHash{mtk8}~~\textcolor{mtk8}{ax.annotate("Lose:}~~\mtkPercent{mtk8}~~\textcolor{mtk8}{d" }~~\mtkPercent{mtk8}~~\textcolor{mtk8}{ countlose, xy = (dt.strpt}~~\textcolor{mtk8}{ime('2019-10-29 17:30','}~~\mtkPercent{mtk8}~~\textcolor{mtk8}{Y-}~~\mtkPercent{mtk8}~~\textcolor{mtk8}{m-}~~\mtkPercent{mtk8}~~\textcolor{mtk8}{d }~~\mtkPercent{mtk8}~~\textcolor{mtk8}{H:}~~\mtkPercent{mtk8}~~\textcolor{mtk8}{M'), 0), size}~~\textcolor{mtk8}{ = 10, color='r')}~

labels = ax.get_xticklabels()
plt.setp(labels, rotation=~\textcolor{mtk7}{45}~, fontsize=~\textcolor{mtk7}{10}~)

plt.show()
\end{codeCell}
\end{paracol}

\columnratio{0.09}
\begin{paracol}{2}
\begin{cellExecute}[escapechar=~]
~\outputPrompt{11}~
\end{cellExecute}
\switchcolumn
\begin{resultCell}[escapechar=~]
\end{resultCell}
\end{paracol}

\begin{figure}[H]
\centering
\includegraphics[keepaspectratio,width=0.9\hsize]{/Users/_2pt/colaboratoryfigures/fig191130062652_27.png}
\end{figure}
\subsection{時間vs累計で推移をみる}

\columnratio{0.09}
\begin{paracol}{2}
\smallskip
\begin{cellExecute}[escapechar=~]
~\inputPrompt{12}~
\end{cellExecute}
\switchcolumn
\begin{codeCell}[escapechar=~]
y_win=[a[~\textcolor{mtk7}{1}~] ~\textcolor{mtk17}{for}~ a ~\textcolor{mtk6}{in}~ data]
y_win_sum=[~\textcolor{mtk12}{int}~(y_win.pop(~\textcolor{mtk7}{0}~))]
~\textcolor{mtk17}{for}~ yi ~\textcolor{mtk6}{in}~ y_win:
  y_win_sum.append(y_win_sum[~\textcolor{mtk7}{-1}~] + yi)
\end{codeCell}
\end{paracol}

\columnratio{0.09}
\begin{paracol}{2}
\smallskip
\begin{cellExecute}[escapechar=~]
~\inputPrompt{13}~
\end{cellExecute}
\switchcolumn
\begin{codeCell}[escapechar=~]
y_lose=[~\textcolor{mtk7}{1}~^a[~\textcolor{mtk7}{1}~] ~\textcolor{mtk17}{for}~ a ~\textcolor{mtk6}{in}~ data]
y_lose_sum=[~\textcolor{mtk12}{int}~(y_lose.pop(~\textcolor{mtk7}{0}~))]
~\textcolor{mtk17}{for}~ yi ~\textcolor{mtk6}{in}~ y_lose:
  y_lose_sum.append(y_lose_sum[~\textcolor{mtk7}{-1}~] + yi)
\end{codeCell}
\end{paracol}

\columnratio{0.09}
\begin{paracol}{2}
\smallskip
\begin{cellExecute}[escapechar=~]
~\inputPrompt{14}~
\end{cellExecute}
\switchcolumn
\begin{codeCell}[escapechar=~]
y_sum=[i+~\textcolor{mtk7}{1}~ ~\textcolor{mtk17}{for}~ i ~\textcolor{mtk6}{in}~ ~\textcolor{mtk13}{range}~(~\textcolor{mtk13}{len}~(data))]
\end{codeCell}
\end{paracol}

\columnratio{0.09}
\begin{paracol}{2}
\smallskip
\begin{cellExecute}[escapechar=~]
~\inputPrompt{15}~
\end{cellExecute}
\switchcolumn
\begin{codeCell}[escapechar=~]
fig, ax = plt.subplots(figsize=(~\textcolor{mtk7}{8}~, ~\textcolor{mtk7}{8}~))

ax.plot(x,y_win_sum,~\textcolor{mtk25}{"-"}~,label=~\textcolor{mtk25}{"Win"}~,color=~\textcolor{mtk25}{"orange"}~)
ax.plot(x,y_lose_sum,~\textcolor{mtk25}{"-"}~,label=~\textcolor{mtk25}{"Lose"}~)
ax.plot(x,y_sum,~\textcolor{mtk25}{"-"}~,label=~\textcolor{mtk25}{"All"}~,color=~\textcolor{mtk25}{"black"}~)

ax.xaxis.set_major_locator(mdates.HourLocator(interval=~\textcolor{mtk7}{2}~))
ax.xaxis.set_major_formatter(mdates.DateFormatter(~\textcolor{mtk25}{"}~~\mtkPercent{mtk25}~~\textcolor{mtk25}{Y-}~~\mtkPercent{mtk25}~~\textcolor{mtk25}{m-}~~\mtkPercent{mtk25}~~\textcolor{mtk25}{d }~~\mtkPercent{mtk25}~~\textcolor{mtk25}{H:}~~\mtkPercent{mtk25}~~\textcolor{mtk25}{M"}~))
ax.set_xlim([dt.strptime(~\textcolor{mtk25}{'2019-10-29 07:00'}~,~\textcolor{mtk25}{'}~~\mtkPercent{mtk25}~~\textcolor{mtk25}{Y-}~~\mtkPercent{mtk25}~~\textcolor{mtk25}{m-}~~\mtkPercent{mtk25}~~\textcolor{mtk25}{d }~~\mtkPercent{mtk25}~~\textcolor{mtk25}{H:}~~\mtkPercent{mtk25}~~\textcolor{mtk25}{M'}~), dt.strptime(~\textcolor{mtk25}{'2019-10-30 00:00'}~,~\textcolor{mtk25}{'}~~\mtkPercent{mtk25}~~\textcolor{mtk25}{Y-}~~\mtkPercent{mtk25}~~\textcolor{mtk25}{m-}~~\mtkPercent{mtk25}~~\textcolor{mtk25}{d }~~\mtkPercent{mtk25}~~\textcolor{mtk25}{H:}~~\mtkPercent{mtk25}~~\textcolor{mtk25}{M'}~)])

ax.grid(color = ~\textcolor{mtk25}{"green"}~, alpha = ~\textcolor{mtk7}{0.5}~, linestyle = ~\textcolor{mtk25}{"--"}~, linewidth = ~\textcolor{mtk7}{1}~)

ax.set_ylabel(~\textcolor{mtk25}{"Count"}~)
ax.set_xlabel(~\textcolor{mtk25}{"Time"}~)

labels = ax.get_xticklabels()
plt.setp(labels, rotation=~\textcolor{mtk7}{45}~, fontsize=~\textcolor{mtk7}{10}~)

ax.annotate(~\textcolor{mtk25}{"Win:}~~\mtkPercent{mtk25}~~\textcolor{mtk25}{d"}~ % y_win_sum[~\textcolor{mtk7}{-1}~], xy = (dt.strptime(~\textcolor{mtk25}{'2019-10-30 00:00'}~,~\textcolor{mtk25}{'}~~\mtkPercent{mtk25}~~\textcolor{mtk25}{Y-}~~\mtkPercent{mtk25}~~\textcolor{mtk25}{m-}~~\mtkPercent{mtk25}~~\textcolor{mtk25}{d }~~\mtkPercent{mtk25}~~\textcolor{mtk25}{H:}~~\mtkPercent{mtk25}~~\textcolor{mtk25}{M'}~), y_win_sum[~\textcolor{mtk7}{-1}~]), size = ~\textcolor{mtk7}{10}~, color=~\textcolor{mtk25}{'r'}~)
ax.annotate(~\textcolor{mtk25}{"Lose:}~~\mtkPercent{mtk25}~~\textcolor{mtk25}{d"}~ % y_lose_sum[~\textcolor{mtk7}{-1}~], xy = (dt.strptime(~\textcolor{mtk25}{'2019-10-30 00:00'}~,~\textcolor{mtk25}{'}~~\mtkPercent{mtk25}~~\textcolor{mtk25}{Y-}~~\mtkPercent{mtk25}~~\textcolor{mtk25}{m-}~~\mtkPercent{mtk25}~~\textcolor{mtk25}{d }~~\mtkPercent{mtk25}~~\textcolor{mtk25}{H:}~~\mtkPercent{mtk25}~~\textcolor{mtk25}{M'}~), y_lose_sum[~\textcolor{mtk7}{-1}~]), size = ~\textcolor{mtk7}{10}~, color=~\textcolor{mtk25}{'r'}~)
ax.annotate(~\textcolor{mtk25}{"All:}~~\mtkPercent{mtk25}~~\textcolor{mtk25}{d"}~ % y_sum[~\textcolor{mtk7}{-1}~], xy = (dt.strptime(~\textcolor{mtk25}{'2019-10-30 00:00'}~,~\textcolor{mtk25}{'}~~\mtkPercent{mtk25}~~\textcolor{mtk25}{Y-}~~\mtkPercent{mtk25}~~\textcolor{mtk25}{m-}~~\mtkPercent{mtk25}~~\textcolor{mtk25}{d }~~\mtkPercent{mtk25}~~\textcolor{mtk25}{H:}~~\mtkPercent{mtk25}~~\textcolor{mtk25}{M'}~), y_sum[~\textcolor{mtk7}{-1}~]), size = ~\textcolor{mtk7}{10}~, color=~\textcolor{mtk25}{'r'}~)


ax.legend()
plt.show()
\end{codeCell}
\end{paracol}

\columnratio{0.09}
\begin{paracol}{2}
\begin{cellExecute}[escapechar=~]
~\outputPrompt{15}~
\end{cellExecute}
\switchcolumn
\begin{resultCell}[escapechar=~]
\end{resultCell}
\end{paracol}

\begin{figure}[H]
\centering
\includegraphics[keepaspectratio,width=0.75\hsize]{/Users/_2pt/colaboratoryfigures/fig191130062652_32.png}
\end{figure}
\par グラフを見てみると
\begin{itemize}

\item はじめのうちは当たり:外れ=2:1くらいで出ている
\item 中盤から当たりが外れに比べてだいぶ出やすくなった
\item \textbf{終盤は出せる当たりが尽きたためか?はずれしかでなくなっている}

\end{itemize}
\par ことがわかります。\textbf{なんだか露骨でおもしろい!}
